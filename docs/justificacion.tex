Mientras nuestras habilidades y capacidades para colectar y guardar datos crece a un ritmo acelerado, nuestras habilidades para procesarlos y analizarlos no lo hace al mismo ritmo, sin hablar de hacerlo en tiempo real \citep{Keim2006, Liao2014}. La AV se utiliza para resolver problemáticas de varios tipos, pero rara vez se asocia el componente espacio-temporal con el componente multidimensional. También, aunque varios trabajos de investigación están enfocados en datos meteorológicos y ambientales, estas aproximaciones están más enfocadas a visualizaciones científicas que a visualizaciones informativas \citep{Liao2014}.

La ciencia de la AV puede reducir el costo de completar un análisis de la calidad del aire al realizar una combinación entre las técnicas de interacción, visualización y técnicas estadísticas en un ambiente interactivo; esto es debido a que las aproximaciones en minería de datos y visualización están fuertemente involucradas en este proceso \citep{Du2018}.