\citet{Lemke2014}, estudiaron la relación geoespacial de la calidad del aire y eventos de asma severa entre la frontera de Michigan, EU., y Ontario, Canadá. Sus resultados demostraron que existe una relación entre la variación inter-urbana de ambas ciudades, en el año 2008, con eventos adversos de enfermedades respiratorias en al menos 3 códigos postales similares.

Basados en la premisa de que la concentración de los principales contaminantes está fuertemente correlacionada con el lugar correspondiente de las emisiones, \citet{Righini2014} desarrollaron un método para evaluar la representatividad de los sitios de monitoreo mediante el análisis de las variaciones espaciales de las emisiones alrededor de estos sitios. Para esto, realizaron un análisis de la variabilidad horizontal de varios contaminantes ambientales mediante un Sistema de Información Geográfica (SIG) con una función estadística de vencindad.

Con un enfoque dirigido hacia la interpretación espacial y visualización de datos, \citet{Hunova2001} desarrolló un método que pretende servir como herramienta para la caracterización de la calidad del aire usando la menor cantidad de factores posibles. Menciona que identificaron tres factores recomendados: contaminación del aire, ozono a nivel del suelo y deposición atmosférica húmeda. \citet{Aguilo2012}, en su tesis doctoral, presenta una nueva aproximación para la realización de mapas de calidad del aire. El objetivo de este estudio pretende demostrar que, la variable de calidad del aire debe ser tomada en cuenta en los procesos de planificación física o territorial. Desarrolla un modelo geoestadístico para la predicción de la calidad del aire en un territorio, el cual se basa en la interpolación de las medidas de concentración de contaminantes. Para el modelo utiliza \textit{kriging} ordinario con una homogeneización de los datos para eliminar el carácter local y así quitar las tendencias debidas a las variaciones temporales. También desde un punto de vista de desarrollo territorial. \citet{Molina2021} muestran la relación entre la calidad del aire y el desarrollo urbano sostenible. En su estudio. Presentan un análisis de estudios donde desarrollaron modelos de aprendizaje automático para pronosticar el desarrollo urbano sostenible y la calidad del aire. Mencionan que algunas herramientas de aprendizaje automático son usadas primeramente para agrupación y clasificación, mientras que herramientas como redes neuronales artificiales y máquinas de soporte vectorial son utilizadas mayormente para predecir distintos tipos de eventos.

Con respecto a la AV, propiamente, \citet{Liao2014} presentan un sistema de AV basado en la web, donde realizan un análisis de datos la calidad del aire; en este trabajo consideran los atributos espacio-temporales y sus variadas dimensiones para ofrecer tres principales vistas: un mapa usado para análisis espacial, gráficas de dispersión para análisis temporal y un gráfico de coordenadas paralelas para el análisis multidimensional. En un esfuerzo por definir agrupaciones espaciales entre múltiples atributos, \citet{Zhou2017} diseñaron un sistema de AV que usa un escalamiento multidimensional (MDS, por sus siglas en inglés) para transformar los datos de estaciones de monitoreo, de alta dimensionalidad, en gráficas de 2D. Usan el método de clúster jerárquico para obtener agregados espaciales y usan diagramas de Voronoi para permitir a los usuarios visualizar e interactuar con los agregados espaciales. \citet{Du2018} desarrolaron un sistema de AV para analizar la calidad del aire. Su sistema permite analizar y entender el adelanto y retraso de la correlación entre una región objetivo y regiones cercanas, con el fin de explorar relaciones causales potenciales entre las regiones. \citet{Engel2012} realizaron un estudio en donde muestran que la visualización es esencial cuando se trata de proveer los medios para encontrar menores dimensiones, no ambiguas y físicamente correctas a partir de la transformación de los datos de calidad del aire. \citet{Yang2021} diseñaron e implementaron un sistema de análisis visual, así como una variedad de modelos visuales, con el fin de ayudar al usuario a filtrar, escalar, y explorar datos espacio-temporales de contaminación del aire. \citet{Li2016}, desarrollaron una aproximación de múltiples visualizaciones con el fin de encontrar la relación entre el smog en China con atributos meteorológicos; diseñaron una vista de detección de correlación que visualiza simultaneamente los patrones de cambio de la calidad del aire a través de varias ciudades, relacionado con los atributos meteorológicos.
