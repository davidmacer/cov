\subsection{Analítica Visual}\label{sec:analitica_visual}

Existen diferentes definiciones para la Analítica Visual (AV). \cite{Thomas2005}, en su artículo \textit{Illuminating the Path}, la definen como la ciencia del razonamiento analítico, facilitado por interfaces visuales interactivas; mientras que \cite{Keim2008} mencionan que la AV combina técnicas de análisis automatizado con visualizaciones interactivas para un efectivo entendimiento, razonamiento y toma de decisiones, basada en grandes y complejos conjuntos de datos.

Las herramientas y técnicas de la AV permite a los usuarios: 1) sintetizar información y obtener una visión de datos que son masivos, dinámicos, ambíguos y muchas veces conflictivos, 2) detectar lo esperado y descubrir lo inesperado, 3) proveer evaluaciones oportunas, comprensibles y defendibles, y 4) comunicar evaluaciones de forma efectiva para la acción \citep{Keim2008, Thomas2005}.

En general, la visión que impulsa la AV es convertir la sobrecarga de información en una oportunidad. El objetivo de la AV es hacer el procesamiento de datos e información un medio transparente para un discruso analítico. De esta forma, la AV fomentará la evaluación constructiva, la corrección y la mejora rápida de nuestros procesos y modelos y, en úlitma instancia, la mejora de nuestro conocimiento y nuestras decisiones \citep{Keim2008}.

Según \cite{Thomas2005}, la AV es un campo multidisciplinario que incluye las siguientes áreas de enfoque:

\begin{itemize}
    \item \textit{Técnicas de razonamiento analítico} que permiente al usuario obtener visiones profundas que apoyan directamente las evaluaciones, planes y toma de decisiones,
    \item \textit{Representaciones visuales y técnicas interactivas} que aprovechan el ancho de banda del ojo humano y su vía hacia la mente, lo cual permite al usuario ver, explorar y entender grandes cantidades de información de una sola vez,
    \item \textit{Representación y transformación de datos} que convierten todo tipo de datos conflictivos y dinámicos en medios que apoyan la visualización y el análisis, y
    \item Técnicas para apoyar la \textit{producción, presentación y diseminación} de los resultados de un análisis para comunicar la información en un apropiado contexto para una gran variedad de audiencias.
\end{itemize} 

Muchas personas se confunden entre los términos de AV y Visualización de Información, debido a que sí existe cierto traslape entre estas dos áreas, y el trabajo de la visualición de información está altamente relacionado con AV; sin embargo, el trabajo de la visualiación tradicional no necesariamente lidia con tareas de análisis, ni siempre usa algoritmos avanzados para análisis de datos. La AV, más allá de ser solo visualizaciones, es más bien vista como un acercamiento integral para la toma de decisiones, combinando visualizaciones, factores humanos y análisis de datos \citep{Keim2008}.

\cite{Keim2008} mencionan que la AV se construye sobre una variedad de campos de investigación científica. Como base, la AV integra la Visualización Científica y de Información con Tecnologías de Manejo de Datos y Análisis de Datos, así como en Percepción Humana e investigación congnitiva. Mencionan también que, para una investigación efectiva, también requiere de una infraestructura apropiada (en términos de \textit{software}, conjuntos de datos y repositiorios de problemas analíticos relacionados) y para desarrollar una metodología de evaluación confiable.