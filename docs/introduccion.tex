El rápido desarrollo de la sociedad industrial, el desarrollo de las zonas urbanas y el incremento de la concentración de la población han contribuido a la proliferación de problemas ambientales. La contaminación del aire se ha convertido en un problema que ha atraído la atención del público y los gobiernos debido a su impacto en la salud humana y en el desarrollo social \citep{Zhou2017, INECC2017, Deng2020}. Las actividades diarias generadas por las industrias, el comercio y el tránsito vehicular, suelen producir una gran cantidad de sustancias y partículas atmosféricas que modifican la composición natural del aire \citep{Engel2012, INECC2017}. Referente a la salud humana, la contaminación del aire puede causar o contribuir en un incremento en la mortalidad o en serios padecimientos respiratorios y cardiovasculares \citep{Kampa2008, Mathioudakis2020}. 

La representatividad espacial y temporal de las estaciones de monitoreo de la calidad del aire es un tema relevante, pues permite describir los complejos patrones espacio-temporales de la contaminación atmosférica en un área específica; la finalidad de lo anterior radica en alcanzar un control de costo-beneficio de la calidad del aire \citep{Righini2014}. Con el desarrollo y el amplio despliegue de sensores que permiten monitorear la calidad del aire, se generan grandes cantidades de datos con distintas y bien detalladas escalas espacio-temporales \citep{Liao2014, Zhou2017}. Estos complejos conjuntos de datos sobrepasan la habilidad y capacidad de los métodos tradicionales de procesamiento, por lo que muchos investigadores siguen explorando nuevas formas para procesar y analizar estos conjuntos de datos ambientales de forma efectiva \citep{Liao2014}. Así mismo, los procesamientos y análisis de los datos de la calidad del aire, en sus diferentes escalas espacio-temporales, pueden ser de mucha ayuda para que los tomadores de decisiones puedan explorar las causas de la contaminación del aire y encontrar, de esta forma, soluciones óptimas a este problema \citep{Zhou2017}.

La analítica visual (AV), como un campo de investigación emergente y multidisciplinario, se enfoca manejar grandes volúmenes de datos masivos, heterogéneos y dinámicos, y presentarlos en forma de imágenes y gráficas; conecta, de esta forma, la comunicación entre el usuario y los datos para establecer un fuerte modelo mental basado en la capacidad de la cognición humana para encontrar patrones o alguna información oculta en los datos \citep{Liao2014, Keim2006}. La tecnología de la AV, basada en la fusión efectiva de el diseño visual y el modelo de minería de datos, guía de manera interactiva al usuario para analizar y explorar las características de objetos ocultos, procesos y eventos en los datos geoespaciales \citep{Zhou2018}. De esta manera, algunos investigadores introducen la AV en sus análisis de la calidad del aire, obteniendo resultados significativos \citep{Liao2014}. Entonces, se puede decir que el objetivo de la investigación de la AV es convertir la sobrecarga de información en una oportunidad \citep{Keim2006}.
