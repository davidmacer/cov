El SIMAT cuenta con más de 40 estaciones de monitoreo que abarcan 16 alcaldias en la Ciudad de México, 59 municipios del Estado de México y el municipio de Tizayuca, Hidalgo. La base de datos del inventario de emisiones contiene información de las emisiones por categoría, contaminante y entidades que integran la ZMVM. Los contaminantes que se monitorean y se incluyen en la base de datos son:

\FloatBarrier
\begin{table}
    \centering
    \caption{Variables de la calidad del aire que son parte del programa de monitoreo del Sistema de Monitoreo Atmosféricos de la Ciudad de México (SIMAT)}
    \label{tab:aire}
    \begin{tabular}{ | c | c | }
        \hline
        \multicolumn{2}{|c|}{Variables de la calidad del aire} \\
        \hline
        Abreviatura de la variable & Descripción de la variable \\
        \hline
        PM10        & Partículas menores a 10 micrómetros \\ 
        PM2.5       & Partículas menores a 2.5 micrómetros \\  
        PMco        & Partículas coarse \\
        SO$_{2}$    & Dióxido de azufre \\  
        CO          & Monóxido de carbono \\  
        NO          & Óxido nítrico \\
        NO$_{X}$    & Óxidos de nitrógeno \\  
        COT         & Compuestos orgánicos totales \\  
        COV         & Compuestos orgánicos volátiles \\  
        NH$_{3}$    & Amoniaco \\  
        CN          & Carbono negro \\  
        TOX         & Compuestos tóxicos \\  
        CO$_{2}$    & Dióxido de carbono \\  
        CH$_{4}$    & Metano \\  
        N$_{2}$O    & Ócido nitroso \\  
        CO$_{2}$eq. & Dióxido de carbono \\  
        HFC         & Hidrofluorocarbonos \\
        O$_3$       & Ozono \\
        BEN         & Benceno \\
        TOL         & Tolueno \\
        SO$_4$      & Sulfato \\
        NO$_3$      & Nitrato \\
        Cl          & Cloruro \\
        CO$_3$      & Carbonato \\
        H           & Hidrógeno \\
        NH$_4$      & Amonio \\
        Ca          & Calcio \\
        Mg          & Magnesio \\
        Na          & Sodio \\
        K           & Potasio \\
        PST         & Partículas suspendidas totales \\
        PbPST       & Plomo contenido en PST \\
        PbPM        & Plomo contenido en PM10 \\
        ETB         & Etilbenceno \\
        XIL         & Xilenos \\
        \hline
    \end{tabular}
\end{table}
\FloatBarrier

En muchas de las estaciones se realizan también mediciones contínuas de las principales variables meteorológicas de superficie.

\FloatBarrier
\begin{table}
    \centering
    \caption{Variables meteorológicas que son parte del programa de monitoreo del Sistema de Monitoreo Atmosféricos de la Ciudad de México (SIMAT)\label{tab:meteor}}
    \begin{tabular}{ | c | c | }
        \hline
        \multicolumn{2}{|c|}{Variables de la calidad del aire} \\
        \hline
        Abreviatura de la variable & Descripción de la variable \\
        \hline
        WSP         & Velocidad del viento \\
        WDR         & Direccion del viento \\
        TMP         & Temperatura ambiente \\
        RH          & Humedad relativa \\
        UVA         & Radiación UV-A \\
        UVB         & Radiación UV-B \\
        GR          & Radiación global \\
        PAR         & Radiación FA \\
        PA          & Presión Atmosférica \\
        pH          & Potencial de hidrógeno \\
        PP          & Precipitacion pluvial \\
        CE          & Conductividad eléctrica \\
        \hline
    \end{tabular}
\end{table}
\FloatBarrier

Las diferentes variables, tanto de la calidad del aire (tabla \ref{tab:aire}) como meteorológicas (tabla \ref{tab:meteor}), se presentan a diferentes intervalos temporales (1 hr., 3 hr., 8 hr. y 24 hr.)

Se pretende obtener las variables de la calidad del aire y meteorológicas de los últimos años en sus diferentes intervalos temporales con el fin de realizar el procesamiento y analisis para cumplir con el objetivo general y los objetivos específicos.