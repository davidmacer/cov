\documentclass[12pt]{article}
% -------------------------------------------------------------------
% Pacotes básicos
\usepackage[spanish,mexico]{babel}										% Idioma a ser usado
\usepackage{hyperref}
                                                                % Trocar "english" para "brazil" para artigos escritos em língua portuguesa 
\usepackage[utf8]{inputenc}										% Escrita de caracteres acentuados e cedilhas - 1
\usepackage[T1]{fontenc}										% Escrita de caracteres acentuados + outros detalhes técnicos fundamentais
% -------------------------------------------------------------------
% Pacotes para inserção de figuras e subfiguras
\usepackage{subfig,epsfig,tikz,float,placeins}		            % Packages de figuras. 
\graphicspath{ {./imgs/} }
% -------------------------------------------------------------------
% Pacotes para inserção de tabelas
\usepackage{booktabs,multicol,multirow,tabularx,array}          % Packages para tabela
\usepackage[round,sort]{natbib}
% -------------------------------------------------------------------
% Definição de comprimentos
\setlength{\parindent}{0pt}
\setlength{\parskip}{5pt}
\textwidth 13.5cm
\textheight 19.5cm
\columnsep .5cm
% -------------------------------------------------------------------
% Título do seu artigo 
\title{\renewcommand{\baselinestretch}{1.17}\bf%
Análisis de la Dinámica de la Calidad del Aire en la Zona Metropolitana del Valle de México \ 
mediante Analítica Visual \\

\begin{large} 
    Propuesta de anteproyecto para el \ 
    Doctorado en Ciencias de Información Geoespacial
\end{large}
}
% -------------------------------------------------------------------
% Autorias
\author{%
Jorge David Martínez Cervantes\textsuperscript{1} \\ 
\\ 
\normalsize email: \href{mailto:jdmartinez@centrogeo.edu.mx}{jdmartinez@centrogeo.edu.mx} \\
\\ 

Asesor: Dr. Rodrigo Tapia McClung
}

\date{\vspace{-5ex}}
%\date{}  % Toggle commenting to test
% -------------------------------------------------------------------

%Início do documento

\begin{document}

\maketitle

\begin{center}
    {\footnotesize 
    \textsuperscript{1}Centro de Investigación en Ciencias de Información Geoespacial, A.C. \\
    }
\end{center}

\newpage

\tableofcontents

\newpage

\begin{abstract}
	
    \textbf{Palabras clave:} visual analytics, spatio-temporal visualization, time series visualization, multi-dimensional visualization, air pollution \\

\end{abstract}
%----

% -------------------------------------------------------------------
\newpage

\section{Introducción}\label{sec:introduccion}

% El rápido desarrollo de la sociedad industrial, el desarrollo de las zonas urbanas y el incremento de la concentración de la población han contribuido a la proliferación de problemas ambientales. La contaminación del aire se ha convertido en un problema que ha atraído la atención del público y los gobiernos debido a su impacto en la salud humana y en el desarrollo social \citep{Zhou2017, INECC2017, Deng2020}. Las actividades diarias generadas por las industrias, el comercio y el tránsito vehicular, suelen producir una gran cantidad de sustancias y partículas atmosféricas que modifican la composición natural del aire \citep{Engel2012, INECC2017}. Referente a la salud humana, la contaminación del aire puede causar o contribuir en un incremento en la mortalidad o en serios padecimientos respiratorios y cardiovasculares \citep{Kampa2008, Mathioudakis2020}. 

La representatividad espacial y temporal de las estaciones de monitoreo de la calidad del aire es un tema relevante, pues permite describir los complejos patrones espacio-temporales de la contaminación atmosférica en un área específica; la finalidad de lo anterior radica en alcanzar un control de costo-beneficio de la calidad del aire \citep{Righini2014}. Con el desarrollo y el amplio despliegue de sensores que permiten monitorear la calidad del aire, se generan grandes cantidades de datos con distintas y bien detalladas escalas espacio-temporales \citep{Liao2014, Zhou2017}. Estos complejos conjuntos de datos sobrepasan la habilidad y capacidad de los métodos tradicionales de procesamiento, por lo que muchos investigadores siguen explorando nuevas formas para procesar y analizar estos conjuntos de datos ambientales de forma efectiva \citep{Liao2014}. Así mismo, los procesamientos y análisis de los datos de la calidad del aire, en sus diferentes escalas espacio-temporales, pueden ser de mucha ayuda para que los tomadores de decisiones puedan explorar las causas de la contaminación del aire y encontrar, de esta forma, soluciones óptimas a este problema \citep{Zhou2017}.

La analítica visual (AV), como un campo de investigación emergente y multidisciplinario, se enfoca manejar grandes volúmenes de datos masivos, heterogéneos y dinámicos, y presentarlos en forma de imágenes y gráficas; conecta, de esta forma, la comunicación entre el usuario y los datos para establecer un fuerte modelo mental basado en la capacidad de la cognición humana para encontrar patrones o alguna información oculta en los datos \citep{Liao2014, Keim2006}. La tecnología de la AV, basada en la fusión efectiva de el diseño visual y el modelo de minería de datos, guía de manera interactiva al usuario para analizar y explorar las características de objetos ocultos, procesos y eventos en los datos geoespaciales \citep{Zhou2018}. De esta manera, algunos investigadores introducen la AV en sus análisis de la calidad del aire, obteniendo resultados significativos \citep{Liao2014}. Entonces, se puede decir que el objetivo de la investigación de la AV es convertir la sobrecarga de información en una oportunidad \citep{Keim2006}.


\section{Antecedentes y estado del arte}\label{sec:antecedentes}

% \citet{Lemke2014}, estudiaron la relación geoespacial de la calidad del aire y eventos de asma severa entre la frontera de Michigan, EU., y Ontario, Canadá. Sus resultados demostraron que existe una relación entre la variación inter-urbana de ambas ciudades, en el año 2008, con eventos adversos de enfermedades respiratorias en al menos 3 códigos postales similares.

Basados en la premisa de que la concentración de los principales contaminantes está fuertemente correlacionada con el lugar correspondiente de las emisiones, \citet{Righini2014} desarrollaron un método para evaluar la representatividad de los sitios de monitoreo mediante el análisis de las variaciones espaciales de las emisiones alrededor de estos sitios. Para esto, realizaron un análisis de la variabilidad horizontal de varios contaminantes ambientales mediante un Sistema de Información Geográfica (SIG) con una función estadística de vencindad.

Con un enfoque dirigido hacia la interpretación espacial y visualización de datos, \citet{Hunova2001} desarrolló un método que pretende servir como herramienta para la caracterización de la calidad del aire usando la menor cantidad de factores posibles. Menciona que identificaron tres factores recomendados: contaminación del aire, ozono a nivel del suelo y deposición atmosférica húmeda. \citet{Aguilo2012}, en su tesis doctoral, presenta una nueva aproximación para la realización de mapas de calidad del aire. El objetivo de este estudio pretende demostrar que, la variable de calidad del aire debe ser tomada en cuenta en los procesos de planificación física o territorial. Desarrolla un modelo geoestadístico para la predicción de la calidad del aire en un territorio, el cual se basa en la interpolación de las medidas de concentración de contaminantes. Para el modelo utiliza \textit{kriging} ordinario con una homogeneización de los datos para eliminar el carácter local y así quitar las tendencias debidas a las variaciones temporales. También desde un punto de vista de desarrollo territorial. \citet{Molina2021} muestran la relación entre la calidad del aire y el desarrollo urbano sostenible. En su estudio. Presentan un análisis de estudios donde desarrollaron modelos de aprendizaje automático para pronosticar el desarrollo urbano sostenible y la calidad del aire. Mencionan que algunas herramientas de aprendizaje automático son usadas primeramente para agrupación y clasificación, mientras que herramientas como redes neuronales artificiales y máquinas de soporte vectorial son utilizadas mayormente para predecir distintos tipos de eventos.

Con respecto a la AV, propiamente, \citet{Liao2014} presentan un sistema de AV basado en la web, donde realizan un análisis de datos la calidad del aire; en este trabajo consideran los atributos espacio-temporales y sus variadas dimensiones para ofrecer tres principales vistas: un mapa usado para análisis espacial, gráficas de dispersión para análisis temporal y un gráfico de coordenadas paralelas para el análisis multidimensional. En un esfuerzo por definir agrupaciones espaciales entre múltiples atributos, \citet{Zhou2017} diseñaron un sistema de AV que usa un escalamiento multidimensional (MDS, por sus siglas en inglés) para transformar los datos de estaciones de monitoreo, de alta dimensionalidad, en gráficas de 2D. Usan el método de clúster jerárquico para obtener agregados espaciales y usan diagramas de Voronoi para permitir a los usuarios visualizar e interactuar con los agregados espaciales. \citet{Du2018} desarrolaron un sistema de AV para analizar la calidad del aire. Su sistema permite analizar y entender el adelanto y retraso de la correlación entre una región objetivo y regiones cercanas, con el fin de explorar relaciones causales potenciales entre las regiones. \citet{Engel2012} realizaron un estudio en donde muestran que la visualización es esencial cuando se trata de proveer los medios para encontrar menores dimensiones, no ambiguas y físicamente correctas a partir de la transformación de los datos de calidad del aire. \citet{Yang2021} diseñaron e implementaron un sistema de análisis visual, así como una variedad de modelos visuales, con el fin de ayudar al usuario a filtrar, escalar, y explorar datos espacio-temporales de contaminación del aire. \citet{Li2016}, desarrollaron una aproximación de múltiples visualizaciones con el fin de encontrar la relación entre el smog en China con atributos meteorológicos; diseñaron una vista de detección de correlación que visualiza simultaneamente los patrones de cambio de la calidad del aire a través de varias ciudades, relacionado con los atributos meteorológicos.


\section{Justificación}\label{sec:justificacion}

% Mientras nuestras habilidades y capacidades para colectar y guardar datos crece a un ritmo acelerado, nuestras habilidades para procesarlos y analizarlos no lo hace al mismo ritmo, sin hablar de hacerlo en tiempo real \citep{Keim2006, Liao2014}. La AV se utiliza para resolver problemáticas de varios tipos, pero rara vez se asocia el componente espacio-temporal con el componente multidimensional. También, aunque varios trabajos de investigación están enfocados en datos meteorológicos y ambientales, estas aproximaciones están más enfocadas a visualizaciones científicas que a visualizaciones informativas \citep{Liao2014}.

La ciencia de la AV puede reducir el costo de completar un análisis de la calidad del aire al realizar una combinación entre las técnicas de interacción, visualización y técnicas estadísticas en un ambiente interactivo; esto es debido a que las aproximaciones en minería de datos y visualización están fuertemente involucradas en este proceso \citep{Du2018}.

\section{Marco Teórico}\label{sec:marco_teorico}

\subsection{Analítica Visual}\label{sec:analitica_visual}

Existen diferentes definiciones para la Analítica Visual (AV). \cite{Thomas2005}, en su artículo \textit{Illuminating the Path}, la definen como la ciencia del razonamiento analítico, facilitado por interfaces visuales interactivas; mientras que \cite{Keim2008} mencionan que la AV combina técnicas de análisis automatizado con visualizaciones interactivas para un efectivo entendimiento, razonamiento y toma de decisiones, basada en grandes y complejos conjuntos de datos.

Las herramientas y técnicas de la AV permite a los usuarios: 1) sintetizar información y obtener una visión de datos que son masivos, dinámicos, ambíguos y muchas veces conflictivos, 2) detectar lo esperado y descubrir lo inesperado, 3) proveer evaluaciones oportunas, comprensibles y defendibles, y 4) comunicar evaluaciones de forma efectiva para la acción \citep{Keim2008, Thomas2005}.

En general, la visión que impulsa la AV es convertir la sobrecarga de información en una oportunidad. El objetivo de la AV es hacer el procesamiento de datos e información un medio transparente para un discruso analítico. De esta forma, la AV fomentará la evaluación constructiva, la corrección y la mejora rápida de nuestros procesos y modelos y, en úlitma instancia, la mejora de nuestro conocimiento y nuestras decisiones \citep{Keim2008}.

Según \cite{Thomas2005}, la AV es un campo multidisciplinario que incluye las siguientes áreas de enfoque:

\begin{itemize}
    \item \textit{Técnicas de razonamiento analítico} que permiente al usuario obtener visiones profundas que apoyan directamente las evaluaciones, planes y toma de decisiones,
    \item \textit{Representaciones visuales y técnicas interactivas} que aprovechan el ancho de banda del ojo humano y su vía hacia la mente, lo cual permite al usuario ver, explorar y entender grandes cantidades de información de una sola vez,
    \item \textit{Representación y transformación de datos} que convierten todo tipo de datos conflictivos y dinámicos en medios que apoyan la visualización y el análisis, y
    \item Técnicas para apoyar la \textit{producción, presentación y diseminación} de los resultados de un análisis para comunicar la información en un apropiado contexto para una gran variedad de audiencias.
\end{itemize} 

Muchas personas se confunden entre los términos de AV y Visualización de Información, debido a que sí existe cierto traslape entre estas dos áreas, y el trabajo de la visualición de información está altamente relacionado con AV; sin embargo, el trabajo de la visualiación tradicional no necesariamente lidia con tareas de análisis, ni siempre usa algoritmos avanzados para análisis de datos. La AV, más allá de ser solo visualizaciones, es más bien vista como un acercamiento integral para la toma de decisiones, combinando visualizaciones, factores humanos y análisis de datos \citep{Keim2008}.

\cite{Keim2008} mencionan que la AV se construye sobre una variedad de campos de investigación científica. Como base, la AV integra la Visualización Científica y de Información con Tecnologías de Manejo de Datos y Análisis de Datos, así como en Percepción Humana e investigación congnitiva. Mencionan también que, para una investigación efectiva, también requiere de una infraestructura apropiada (en términos de \textit{software}, conjuntos de datos y repositiorios de problemas analíticos relacionados) y para desarrollar una metodología de evaluación confiable.

\section{Planteamiento del Problema}\label{sec:planteamiento}

% Esta propuesta plantea analizar la calidad del aire en la Zona Metropolitana del Valle de México mediante una metodología de analítica visual. Se pretende crear un sistema de analítica visual, basado en web, que guíe al usuario en el análisis de los datos. El sistema estará diseñado tomando en cuenta: modelos de minería de datos de calidad del aire y datos atmosféricos; modelos visuales para interacción con el usuario; diferentes escalas espacio-temporales; y tomará en cuenta la multidimensionalidad de los datos para proporcionar reducciones de dimensión estadísticamente y físicamente correctas.

\section{Hipótesis o preguntas de investigación}\label{sec:hipotesis}

% \begin{itemize}
    \item ¿Cuál es la relación existente entre los datos de calidad del aire de la ZMVM y los atributos meteorológicos?
    \item ¿Cuáles son los mejores métodos para realizar una reducción de dimensionalidad de los atributos de la calidad del aire?
    \item ¿Cuál es la mejor forma de representar visualmente los diferentes atributos (variables) de la calidad del aire de las diferentes estaciones de monitoreo?
    \item ¿Qué tipo de patrones espacio-temporales existen en la calidad del aire de la ZMVM en un tiempo determinado?
    \item ¿Cómo se puede representar la evolución de los patrones espaciales a través del tiempo?
    \item ¿Cómo se puede representar la evolución de los patrones espaciales a través de diferentes escalas temporales?
\end{itemize}

\section{Objetivo General}\label{sec:objetivos}

% El objetivo general de la propuesta es crear un sistema de analítica visual basado en web para la gestión, procesamiento y análisis de datos de calidad del aire en la ZMVM y datos meteorológicos, que sea eficiente para los diferentes usuarios y útil para la toma de decisiones.

\subsection{Objetivos particulares}

\begin{itemize}
    \item Crear un modelo de minería de datos para datos de calidad del aire y datos meteorológicos.
    \item Determinar los mejores métodos para realizar una reducción de dimensionalidad de los datos de la calidad del aire.
    \item Determinar patrones de agrupación espacial de los datos.
    \item Determinar la evolución de las agrupaciones espaciales en las diferentes escalas temporales.
    \item Crear modelos de visualización para los atributos de la calidad del aire y los atributos meteorológicos.
\end{itemize}

\section{Diseño Metodológico}\label{metodologia}

% El SIMAT cuenta con más de 40 estaciones de monitoreo que abarcan 16 alcaldias en la Ciudad de México, 59 municipios del Estado de México y el municipio de Tizayuca, Hidalgo. La base de datos del inventario de emisiones contiene información de las emisiones por categoría, contaminante y entidades que integran la ZMVM. Los contaminantes que se monitorean y se incluyen en la base de datos son:

\FloatBarrier
\begin{table}
    \centering
    \caption{Variables de la calidad del aire que son parte del programa de monitoreo del Sistema de Monitoreo Atmosféricos de la Ciudad de México (SIMAT)}
    \label{tab:aire}
    \begin{tabular}{ | c | c | }
        \hline
        \multicolumn{2}{|c|}{Variables de la calidad del aire} \\
        \hline
        Abreviatura de la variable & Descripción de la variable \\
        \hline
        PM10        & Partículas menores a 10 micrómetros \\ 
        PM2.5       & Partículas menores a 2.5 micrómetros \\  
        PMco        & Partículas coarse \\
        SO$_{2}$    & Dióxido de azufre \\  
        CO          & Monóxido de carbono \\  
        NO          & Óxido nítrico \\
        NO$_{X}$    & Óxidos de nitrógeno \\  
        COT         & Compuestos orgánicos totales \\  
        COV         & Compuestos orgánicos volátiles \\  
        NH$_{3}$    & Amoniaco \\  
        CN          & Carbono negro \\  
        TOX         & Compuestos tóxicos \\  
        CO$_{2}$    & Dióxido de carbono \\  
        CH$_{4}$    & Metano \\  
        N$_{2}$O    & Ócido nitroso \\  
        CO$_{2}$eq. & Dióxido de carbono \\  
        HFC         & Hidrofluorocarbonos \\
        O$_3$       & Ozono \\
        BEN         & Benceno \\
        TOL         & Tolueno \\
        SO$_4$      & Sulfato \\
        NO$_3$      & Nitrato \\
        Cl          & Cloruro \\
        CO$_3$      & Carbonato \\
        H           & Hidrógeno \\
        NH$_4$      & Amonio \\
        Ca          & Calcio \\
        Mg          & Magnesio \\
        Na          & Sodio \\
        K           & Potasio \\
        PST         & Partículas suspendidas totales \\
        PbPST       & Plomo contenido en PST \\
        PbPM        & Plomo contenido en PM10 \\
        ETB         & Etilbenceno \\
        XIL         & Xilenos \\
        \hline
    \end{tabular}
\end{table}
\FloatBarrier

En muchas de las estaciones se realizan también mediciones contínuas de las principales variables meteorológicas de superficie.

\FloatBarrier
\begin{table}
    \centering
    \caption{Variables meteorológicas que son parte del programa de monitoreo del Sistema de Monitoreo Atmosféricos de la Ciudad de México (SIMAT)\label{tab:meteor}}
    \begin{tabular}{ | c | c | }
        \hline
        \multicolumn{2}{|c|}{Variables de la calidad del aire} \\
        \hline
        Abreviatura de la variable & Descripción de la variable \\
        \hline
        WSP         & Velocidad del viento \\
        WDR         & Direccion del viento \\
        TMP         & Temperatura ambiente \\
        RH          & Humedad relativa \\
        UVA         & Radiación UV-A \\
        UVB         & Radiación UV-B \\
        GR          & Radiación global \\
        PAR         & Radiación FA \\
        PA          & Presión Atmosférica \\
        pH          & Potencial de hidrógeno \\
        PP          & Precipitacion pluvial \\
        CE          & Conductividad eléctrica \\
        \hline
    \end{tabular}
\end{table}
\FloatBarrier

Las diferentes variables, tanto de la calidad del aire (tabla \ref{tab:aire}) como meteorológicas (tabla \ref{tab:meteor}), se presentan a diferentes intervalos temporales (1 hr., 3 hr., 8 hr. y 24 hr.)

Se pretende obtener las variables de la calidad del aire y meteorológicas de los últimos años en sus diferentes intervalos temporales con el fin de realizar el procesamiento y analisis para cumplir con el objetivo general y los objetivos específicos.

\section{Resultados y Discusión}\label{resultados_discusion}

% \input{resultados_discusion}

\section{Conclusión}\label{conclusion}

% \input{conclusion}

\bibliographystyle{abbrvnat}
\bibliography{./docs/biblio}

\end{document}