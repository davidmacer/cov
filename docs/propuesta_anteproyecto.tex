\documentclass[10pt]{article}
% -------------------------------------------------------------------
% Pacotes básicos
\usepackage[spanish]{babel}										% Idioma a ser usado
                                                                % Trocar "english" para "brazil" para artigos escritos em língua portuguesa 
\usepackage[utf8]{inputenc}										% Escrita de caracteres acentuados e cedilhas - 1
\usepackage[T1]{fontenc}										% Escrita de caracteres acentuados + outros detalhes técnicos fundamentais
% -------------------------------------------------------------------
% Pacotes para inserção de figuras e subfiguras
\usepackage{subfig,epsfig,tikz,float}		            % Packages de figuras. 
% -------------------------------------------------------------------
% Pacotes para inserção de tabelas
\usepackage{booktabs,multicol,multirow,tabularx,array}          % Packages para tabela

\usepackage{natbib}
% -------------------------------------------------------------------
% Definição de comprimentos
\setlength{\parindent}{0pt}
\setlength{\parskip}{5pt}
\textwidth 13.5cm
\textheight 19.5cm
\columnsep .5cm
% -------------------------------------------------------------------
% Título do seu artigo 
\title{\renewcommand{\baselinestretch}{1.17}\bf%
Estimación de los componentes orgánicos volátiles en la Zona Metropolitana del Valle de México
}
% -------------------------------------------------------------------
% Autorias
\author{%
Jorge David Martínez Cervantes
}
% -------------------------------------------------------------------

%Início do documento

\begin{document}

\maketitle

\begin{center}
    {\footnotesize 
    Centro de Investigación en Ciencias de Información Geoespacial, A.C. \\
    }
\end{center}
    
    % Here is the abstract.
\begin{abstract}
	Here is where I would say what is in this document.
\end{abstract}
%----

\vspace{-0.5cm}

% -------------------------------------------------------------------

\section{Introducción}\label{sec:introduccion}

Volatile organic compounds (VOCs) play important roles in the tropospheric atmosphere. The oxidation of VOCs by various oxidants contributes to the formation of ground-level ozone (O3) and secondary organic aerosols (SOAs; Louie et al., 2013; Ran et al., 2011), influencing both regional air quality and climate change (Fry et al., 2014) \cite{Wu2020}.

Volatile organic compounds (VOCs) are important precursors to the formation of ground-level ozone, and hence photochemical smog (IPCC, 2007) \cite{Louie2013}.

Volatile organic compounds (VOC) means any compound of carbon, excluding carbon monoxide, carbon dioxide, carbonic acid, metallic carbides or carbonates and ammonium carbonate, which participates in atmospheric photochemical reactions, except those designated by EPA as having negligible photochemical reactivity \cite{EPA2017}.

Volatile organic compounds, or VOCs are organic chemical compounds whose composition makes it possible for them to evaporate under normal indoor atmospheric conditions of temperature and pressure \cite{EPA2017}.

Volatile organic compounds (VOCs) are a broad group of organic compounds that we are exposed to on a daily basis. Commonplace items in our homes such as building materials, paints, furniture, cleaning products, and cosmetics are potential sources of VOCs. Such products emit VOCs as gases that we subsequently inhale. Given that hundred of VOCs may be present in any environment, the exact health effects of this exposure are still unknown. Nevertheless, current evidence suggests that a substantial number of these VOCs can cause adverse health effects, including sensory irritation, respiratory symptoms, and even cancer \cite{Rumchev2007}.

The VOCs are the most prevalent of indoor air pollutants and are defined as having a boiling point ranging between 50°C and 260°C (/16/ in  \cite{Rumchev2007}).

A VOC is any organic compound having an initial boiling point less than or equal to 250° C measured at a standard atmospheric pressure of 101.3 kPa \cite{EPA2017}.

VOCs are sometimes categorized by the ease they will be emitted. For example, the World Health Organization (WHO) categorizes indoor organic pollutants as \cite{EPA2017}:

\begin{itemize}
    \item Very volatile organic compounds (VVOCs)
    \item Volatile organic compounds (VOCs)
    \item Semi-volatile organic compounds (SVOCs)
\end{itemize}

Large quantities of VOCs are emitted into the troposphere from anthropogenic and biogenic sources (World Meteorological Organization, 1995; Guenther et al., 1995,2000; Hein et al., 1997; Sawyer et al., 2000; Placet et al., 2000) \cite{Atkinson2000}.

Emissions of oxides of nitrogen (NO$_{x}$ = NO + NO$_{2}$),volatile organic compounds (VOCs) and sulfur compounds (including SO$_{2}$ and reduced sulfur compounds) lead to a complex series of chemical and physical transformations which result in such e!ects as the formation of ozone in urban and regional areas (National Research Council, 1991) \cite{Atkinson2000}.

Large quantities of VOCs are emitted into the troposphere from anthropogenic and biogenic sources (World Meteorological Organization, 1995; Guenther et al., 1995,2000; Hein et al., 1997; Sawyer et al., 2000; Placet et al., 2000) \cite{Atkinson2000}.

Non-methane organic compounds (NMOC) are also emitted into the troposphere from a variety of anthropogenic sources, including combustion sources (vehicle and fossil-fueled power plant emissions), fuel storage and transport, solvent usage, emissions from industrial operations, land"lls, and hazardous waste facilities (Sawyer et al., 2000; Placet et al., 2000) \cite{Atkinson2000}.

As key precursors of O$_{3}$ and secondary organic aerosols (SOA), (important constituents of PM$_{2.5}$), volatile organic compounds (VOCs) are composed of hundreds of species, which are directly emitted into the atmosphere from a variety of natural and anthropogenic sources \cite{Guo2017}.

The major anthropogenic emission sources of VOCs include vehicular exhaust, fuel evaporation, industrial processes, household products and solvent usage (Guo et al., 2011) \cite{Guo2017}.

\section{Antecedentes}\label{sec:antecedentes}

\section{Estado del Arte}\label{sec:estado_arte}

\section{Justificación}\label{sec:justificacion}

\section{Marco Teórico}\label{sec:marco_teorico}

\section{Planteamiento del Problema}\label{sec:planteamiento}

\section{Hipótesis}\label{sec:hipotesis}

\section{Objetivo General y particulares}\label{sec:objetivos}

\section{Diseño Metodológico}\label{metodologia}

\section{Cronograma de Actividades}\label{sec:cronograma}

\bibliographystyle{plain}
\bibliography{./docs/biblio}

\end{document}