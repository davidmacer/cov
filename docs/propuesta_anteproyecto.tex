\documentclass[10pt]{article}
% -------------------------------------------------------------------
% Pacotes básicos
\usepackage[spanish]{babel}										% Idioma a ser usado
                                                                % Trocar "english" para "brazil" para artigos escritos em língua portuguesa 
\usepackage[utf8]{inputenc}										% Escrita de caracteres acentuados e cedilhas - 1
\usepackage[T1]{fontenc}										% Escrita de caracteres acentuados + outros detalhes técnicos fundamentais
% -------------------------------------------------------------------
% Pacotes para inserção de figuras e subfiguras
\usepackage{subfig,epsfig,tikz,float}		            % Packages de figuras. 
% -------------------------------------------------------------------
% Pacotes para inserção de tabelas
\usepackage{booktabs,multicol,multirow,tabularx,array}          % Packages para tabela

\usepackage{natbib}
% -------------------------------------------------------------------
% Definição de comprimentos
\setlength{\parindent}{0pt}
\setlength{\parskip}{5pt}
\textwidth 13.5cm
\textheight 19.5cm
\columnsep .5cm
% -------------------------------------------------------------------
% Título do seu artigo 
\title{\renewcommand{\baselinestretch}{1.17}\bf%
Estimación de los componentes orgánicos volátiles en la Zona Metropolitana del Valle de México
}
% -------------------------------------------------------------------
% Autorias
\author{%
Jorge David Martínez Cervantes
}
% -------------------------------------------------------------------

%Início do documento

\begin{document}

\date{\today}

\maketitle

\vspace{-0.5cm}

\begin{center}
{\footnotesize 
Centro de Investigación en Ciencias de Información Geoespacial, A.C. \\
}
\end{center}

% -------------------------------------------------------------------

\section{Introducción}\label{sec:1}

\textbf{¿Qué son los compuestos orgánicos volátiles?}

Volatile organic compounds (VOC) means any compound of carbon, excluding carbon monoxide, carbon dioxide, carbonic acid, metallic carbides or carbonates and ammonium carbonate, which participates in atmospheric photochemical reactions, except those designated by EPA as having negligible photochemical reactivity \cite{epavoc}.

\bf{¿Cómo afectal la fotoquímica de manera local, regional y global?}

\bf{¿Cuál es su relación con el cambio climático?}

\section{Antecedentes}\label{sec:2}


\section{Justificación}\label{sec:3}

\bibliographystyle{plain}
\bibliography{./docs/biblio}

\end{document}